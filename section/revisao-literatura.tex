\section{Revisão da Literatura}


%»»» Contextualização » Mercado de Moradias
O mercado de moradias sempre foi motivo de atenção por parte da academia, seja por ser o maior ativo no portfólio das famílias e maior despesa no orçamento familiar, assim como, por sua importância relativa as variáveis macroeconômicas, tais como: emprego, renda e atividade econômica. E com a grande Crise Financeira Imobiliário de 2008 no mercado dos Estados Unidos, e consequente Crise Financeira Mundial,  questões como o que houve? por que aconteceu? ou como evitar o que ocorreu? estiveram na agenda de pesquisa de forma intensa \cite{Davis2015}. 

Todavia, segundo \citeonline{Zezza2007}, o mercado de moradia tem sido analisado, de uma forma geral, por meio de modelos de equilíbrio parcial ou por meio de abordagens espacias ou geográficas. Sendo assim, no que diz respeito a Teoria Econômica, podemos verificar dois grandes campos de estudo que discutem o mercado de moradia em uma perspectiva econômico-financeira, qual seja: o Mercado Habitacional e o Mercado Imobiliário. %\footnote{Estudos dentro do arcabouço da Economia Urbana que investigam a dinâmica de organização, estrutura e desenvolvimento da cidade seria uma \textit{terceira} abordagem de investigação da moradia como objeto de estudo, no entanto, não será tratado no presente projeto por utilizar como variável chave o espaço no qual as relações econômicas são desenvolvidas.}

% resumindo o texto
% Com o aporte teórico bem desenvolvido temos as Teorias da Localização no qual apresenta modelos teóricos com diversas aplicações empíricas, dentre os quais, destacam-se o modelo de \citeonline{VonThunen1826} como um texto seminal, passando pelo seu aprofundamento a partir dos estudos de \citeonline{Alonso1964}, \citeonline{muth1969cities}, \citeonline{Mills1972studies}, onde analisam o desenvolvimento da cidade em uma perspectiva monocêntrica. Posteriormente, modelos do tipo policêntrica, tais como \citeonline{Fujita1982}, \citeonline{ANAS1998} até diversos outros estudos mais recentes no qual se destacam as pesquisas sobre precificação via preços hedônicos.

% resumindo o texto
% Nesse sentido, esses estudos estariam circunscritos, em um sentido amplo, dentro do arcabouço da Economia Urbana ou da Microeconomia Aplicada dedicados em investigar a dinâmica de organização, estrutura e desenvolvimento tendo como variável explícita o espaço no qual a cidade se desenvolve \cite{Osullivan2007}.

% resumindo o texto
% Para \citeonline{Nadalin2011}, mesmo fora da chamada Economia Urbana e das Teorias da Localização, mas ainda dentro da abordagem da Teoria Econômica, que discute a moradia como objeto de estudo, teríamos as Teorias sobre o Mercado Habitacional e sobre o Mercado Imobiliário. 

%»»» Mercado Imobiliário

Para as Teorias do Mercado Habitacional, tratam a moradia por meio de uma abordagem Econômica, investigam a habitação como um bem de consumo a partir de suas características intrínsecas como:  durabilidade, estoque, heterogeneidade, estrutura habitacional como formas de determinação do preço em um mercado no qual, mesmo com características específicas, concorre no orçamento familiar com outros bens \cite{Blank2014struture}.

%»»» Mercado Imobiliário

Já o Mercado Imobiliário, conforme é colocado por \citeonline{Wheaton1990housing}, alguns fatos estilizados são importantes como forma de descrever o mercado imobiliário. O primeiro, diz respeito ao fato de ser um mercado do tipo fluxo-estoque, sendo que no curto prazo seria relativamente fixo, apresentando forte assimetria devido a durabilidade do bem, onde a oferta seria inelásticas a choques negativos. O segundo fato estilizado, o tempo gasto com a venda é muito elevado tornando altos os custo de busca e incertezas por parte do comprador. Por fim, apresenta oscilações provocadas pelas mudanças nos níveis de atividade econômica.

% resumindo o texto
% Desse modo, para \citeonline{Wheaton1990housing} que desenvolve um modelo para o mercado imobiliário que leva em consideração vacância e custos de busca, define que o mercado imobiliário não seria socialmente ótimo, pois teria como característica a informação imperfeita e altos custos de transação. Sendo, portanto, necessário a atuação do governo como forma de minimizar as falhas de mercados existentes.

O Mercado Imobiliário, nesses termos, teria fundamentalmente uma abordagem financista onde a moradia seria tratada como um ativo financeiro, ou seja, seria um investimento dentro das escolhas de portfólio das famílias, onde os mecanismos de ajustes entre oferta e demanda se daria via um volume de estoque e parque tecnológicos fixos no curto prazo.

% resumindo o texto
% Como podemos observar, enquanto as Teorias da Localização tem como objeto explícito a relação da habitação com o espaço, por outro lado, as Teorias da Habitação e Imobiliária, mesmo levando em consideração variáveis espacias, tem como foco o estudo da moradia como uma \textit{commodity} ou como um ativo a partir de variáveis econômico-financeira. Assim posto, dados os objetivos apresentados, nosso recorte metodológico será o Mercado Habitacional e Imobiliário.

Nesse sentido, integrar as Teorias do Mercado Habitacional com os atributos da Teoria do Mercado Imobiliário pode ser um campo bastante frutífero de análise e discussões, pois,  investigar a produção e o consumo da moradia sob a dominância do capitalismo financeiro  estaria de acordo com a nova lógica de produção do próprio espaço urbano contemporâneo \cite{HARVEY2005, ROLNIK2015}. 

Integrar essas circuitos de análise de forma a observar os resultados globais desses mercados, mas que seja microfundamentado não é algo trivial, sobretudo, para o ferramental teórico-metodológico disponível pelo \textit{mainstream} econômico. Porém, por conta do extraordinários avanço dos sistemas computacionais em processar informação, tem crescido de forma acentuada uma abordarem de pesquisa que trata os problemas econômicos por meio do entendimento da economia como um Sistema Complexo \cite{Colander2004}.

%»»» Sistemas Complexos

Nesse sentido, uma abordagem da Teoria Econômica como um Sistema Complexo, parte do entendimento do conceito chave de \enquote{racionalidade limitada} de \citeonline{Simon1980}, onde os problemas econômico são apresentados por meio de \textit{processos emergentes} no qual são constituídos de múltiplas partes que se ligam e integram sem que haja um \textit{controle central} ou analisados simplesmente de uma forma agregativa, isto é, parte-se de estruturas hierárquicas que se adaptam e evoluem de forma \textit{dispersa}, \textit{dinâmica} e \textit{não-linear} do tipo \textit{botton-up} \cite{Foster2005}.

Grande parte desses conceitos e categorias apresentados vem sendo desenvolvidos em pesquisas teóricas e empíricas no campo da Teoria Econômica ao longo do século XX e XXI, demonstrando grande potencial de desenvolvimento e crescimento. Por esse angulo, para \citeonline{Holt2011}, o estudo da Economia como Sistema Complexo seria uma nova era de pensamento econômico.

%»»» Simulação
Dessa maneira, conforme aponta \citeonline{Aziza2016a}, ao tratar os problemas econômicos como um Sistema Complexo tal postura exige uma abordagem própria para o seu desenvolvimento, no qual consiste basicamente na modelagem e simulação computacional. 

Assim, utilizando o \textit{princípio da correspondência} de \citeonline{SAMUELSON1983}, os problemas que não são analiticamente fechados, a simulação computacional pode ser utilizada de forma puramente analística, desde que guarde aderência com a realidade econômica, isto é, quando o pesquisador for fixar parâmetros e condições iniciais deve estabelecer uma correspondência com a realidade estudada, ou ainda via calibração com as informações ou dados efetivamente encontrados.

Nesse sentido, buscaremos calibrar o nosso modelo com dados da realidade, assim, a simulação computacional será tratada nos termos de \citeonline{Ch2014}, ou seja, como complemento ao que se julga conhecer sobre o objeto de estudo via construção de fatos estilizados sobre os mercados estudados, assim, não substituirá as formas metodológicas tradicionais de investigação, pois serão complementares. Nesses termos, adotaremos uma abordagem do tipo \textit{what if?}, no qual se verifica como seria a realidade caso ela tivesse parâmetros ligeiramente alterados por um choque de políticas econômica, por exemplo.

As vantagens da utilização de simulação são inúmeras, conforme aponta  \citeonline{BERGMANN1990}, dentre as quais destacam-se: é uma excelente ferramenta quando se trabalha com problemas de informação e expectativas; dados os resultados recursivos possibilita uma maior aproximação aos fenômenos reais; os resultados não necessitam de um conjunto de pontos de equilíbrio ou de soluções ótimas e em estudos sobre política econômicas os resultados são mais realistas quanto ao efeito de taxas, subsídios ou regulação sobre os agentes, principalmente no que diz respeito ao modo como isso acontece.

%»»»» ABM

Dentro do conjunto de ferramentas disponíveis para construção de modelos de simulação computacional, adotaremos a simulação baseada em agentes (ABM) onde segundo \citeonline{Farias2017}, seria a forma mais adequada de tratamento para simulação em economia a partir de um sistema complexo.

Assim, para \citeonline{AN2012}, em um modelo do tipo ABM temos uma simulação com um dado número de agentes que constituem o sistema que podem ser, por exemplo, compradores, vendedores, empresas, bancos, governo, no qual interagem por meio de regras definidas individualmente e não de uma forma global. Nesse sentido, temos portanto, um processo que emerge das interações de forma autônoma e dinâmica sem que haja uma agregação pura e simples. \citeonline{AXELROD1997}, complementa afirmando que além dos processo emergentes serem dinâmicos são construídos por meio de processos do tipo \textit{botton-up}, ou seja, hierarquicamente distribuídos de baixo para cima.

Em termos do mercado de moradias, os agentes seriam elementos de um sistema econômico que trazem consigo um conjunto de características e comportamentos de produção e consumo que evoluiriam no decorrer da simulação de forma individual no qual as interações emergiriam padrões globais observáveis por meios dos dados de calibragem desse mercado. 

Assim, a modelagem do agente e formulação teórica seria tão simples quanto possível, sendo que a complexidade do modelo seria o resultado dessas interações. Desse forma, conforme assinala \citeonline{Geanakoplos2012Risk}, a utilização da modelagem baseada em agentes como forma de simular o mercado imobiliário tem superado abordagens tradicionais no diz respeito a previsão do comportamento global desses mercados.

% resumindo o texto
Portanto, integrar as Teorias do Mercado Habitacional e Imobiliário em um único modelo categorizado neste projeto como \enquote{Mercado de Moradias} pode ser feita via simulação computacional, pois trata-se de um mercado dinâmico, não-linear e com relações típicas de um Sistema Complexo, ou seja, dadas as relações teóricas e empíricas dos agentes desses mercados, a Modelagem Baseada em Agentes ou \textit{Agent Based Modeling - ABM} pode ser a melhor estratégia para investigar os possíveis resultados observados em diferentes situações, sejam elas espacias, temporais ou em algum ponto dentro do ciclo econômico.
%citar autor sobre ABM como estrategia verificação em diferenes situações, %% verificar ou seria simular 

Por isso, partiremos de uma economia no qual o mercado de moradias é investigado pela abordagem Complexa da Ciência, sendo portanto, nosso guia principal para análises, considerações e principalmente para construção do Modelo Baseado em Agentes que integre, no plano teórico e modelístico, o Mercado Habitacional e Imobiliário de moradias urbanas nas regiões metropolitanas do Brasil.

