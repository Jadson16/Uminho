%-----------------------------------------------------------------------------------------

%                  ÍDEAS E CITAÇÕES 

%-----------------------------------------------------------------------------------------

%%%%%%%%%%%%%%%%%%%%%%%%%%%%%%%%%%%%%%%%%%%%%%%%%%%%%%%%%%%%%%%%%%%%%%%%%%%%%%%%%
%   IDEIAS
%%%%%%%%%%%%%%%%%%%%%%%%%%%%%%%%%%%%%%%%%%%%%%%%%%%%%%%%%%%%%%%%%%%%%%%%%%%%%%%%%
- Índice de Valores de Garantia de Imóveis Residenciais Financiados (IVG-R)

ver sobre esse indicador
	
A construção do indicador é baseada no valor de avaliação de cada imóvel realizada pelo banco no momento da concessão do crédito, e os dados são obtidos do SCR.

tem também o MVG-R

o BACEN não indica o uso desses dados para analise de curto prazo pois pode haver graves distorções analisadas mês a mês.

%-----------------------------------------------------------------------------------------


%%%%%%%%%%%%%%%%%%%%%%%%%%%%%%%%%%%%%%%%%%%%%%%%%%%%%%%%%%%%%%%%%%%%%%%%%%%%%%%%%
% CITAÇÕES
%%%%%%%%%%%%%%%%%%%%%%%%%%%%%%%%%%%%%%%%%%%%%%%%%%%%%%%%%%%%%%%%%%%%%%%%%%%%%%%%%

\section{Simulação em Economia}

%-----------------------------------------------------------------------------------------

% propososta metodolgoica do texto do oreio - projeto de pesquisa

Os modelos macro-dinâmicos de simulação na tradição iniciada por modelo Oreiro \&

Ono possuem características de modelos pós-keynesianos de terceira geração, e, por

conseguinte, não admites solução analítica fechada, por se tratarem de modelos dinâmicos

não-lineares com alto grau de complexidade. Assim, deve-se recorrer à realização de

simulações em computador para a obtenção da dinâmica entre as variáveis endógenas.


Vem a tona, entretanto, uma questão bastante importante, qual seja, como atribuir

valores às condições iniciais e aos parâmetros do modelo. Os autores optaram pela utilização

do método de calibração, o qual é definido, com base em Hansen e Heckman (1996, p.92),

como \textit{um processo de manipulação das variáveis independentes – leia-se aqui os parâmetros

e as condições iniciais – de modo a obter uma combinação plausível entre os dados

observados empiricamente e os resultados simulados.
}

\subsection{Calibração e verificação}

% TEXTO: UMA APLICAÇÃO DE CALIBRAÇÃO DE INCERTEZA PARA ABM
Uma abordagem comum para validar a extensão em que um ABM é realista é através da calibração usando dados históricos. Se feito corretamente, este processo é uma oportunidade para lidar com a incerteza do modelo, que é o tipo específico de incerteza que decorre do fato de que existem muitas variações de um ABM que são plausíveis sob o mesmo conjunto de dados históricos.

A concordância entre modelo e dados não implica que os pressupostos de modelagem descrevam com precisão os processos que produzem o comportamento observado; indica apenas que o modelo é um (talvez vários) que é plausível. Como resultado, lidar com a incerteza do modelo durante a calibração nos permite abordar o trade-off entre um modelo muito flexível que pode facilmente caber nos dados históricos e na incerteza do modelo.

%-----------------------------------------------------------------------------------------
