%%% numeração das páginas
\textual
\pagenumbering{arabic}
\setcounter{page}{2}

\section{Justificativa}

%% INTRODUÇÃO GERAL
A produção de moradias no Brasil apresentou desde da virada do século XXI um período de franca expansão, advindo não somente da estabilidade dos preços, mas também decorrente da financeirização das economias capitalistas e dos ciclos de liquidez internacional do capital, o que se reflete em termos de volume de crédito, número de dades habitacionais financiadas, criação de novos programas, mudança dos arcabouços institucionais. Em particular, observa-se a utilização de políticas monetárias como forma de estimular o setor de construção de civil, seja através de redução de juros, aumento da oferta de novos recursos para serem emprestados pelos bancos e até subsídios do Estado para a população de baixa renda, além da melhora das expectativas econômicas e jurídicas.

Esse movimento de expansão se deu principalmente a partir de 2005 e de forma mais profunda após 2008 (\textit{boom cycle}). No primeiro momento (2005-2008) beneficiado pela conjuntura macroeconômico favorecida pela estabilidade econômica, controle da inflação e queda na taxa de juros. Com a estabilidade econômica e a inflação relativamente controlada os agentes puderam se planejar em um horizonte de tempo maior. No caso específico da queda da taxa de juros sua contribuição se deu pela elevação do saldo em caderneta de poupança, o que mobilizou uma maior oferta de recursos para financiamento habitacional, promovendo desse modo, a diminuição do custo de financiamento, algo fundamental nesse mercado. Em um segundo momento (2008-2013), como medida anticíclica contra a crise financeira internacional, promovida pelo governo federal com articulação dos poderes mais próximos da Presidência – a Casa Civil, chefiada pela então futura Presidente da República, Dilma Rousseff, e pelo Ministério da Fazenda \cite{CARDOSO2015}.

No entanto, a partir de 2014 esse movimento mais geral de expansão demostrou um revez de sua tendência de expansão evidenciando uma retração do setor da construção civil (\textit{bust cycle}), acompanhando o movimento de crise econômica por qual tem passado a economia brasileira desde de 2014. Desse modo, como já apontado por \citeonline{Agnello2018}, percebemos que o ciclo de negócio no mercado imobiliário segue muito bem o ciclo de crescimento da economia como um todo.

%%% Conclusão da justificativa geral
Dessa forma, por se comportar de forma pró-cíclica a produção de moradias desempenha um papel fundamental dentro do circuito econômico de um país, sobretudo em países onde a estrutura social e econômica encontram-se deprimidas, como é o caso do Brasil, pois além de absorver uma grande quantidade de mão de obra, não qualifica inclusive, gera um forte encadeamento de valor em diversos outros setores. Sendo o imóvel também, em um sentido estrito, o \textit{locus} privilegiado da reprodução familiar, onde se consume grande parte da renda gerada pelo agregado familiar.

Dessa forma, faz-se indispensável analisar de forma sistemática o mercado imobiliário brasileiro, pois diante do processo de mundialização do capital \cite{CHESNAIS1996, Chesnais98} a moradia passa a ser demandada também como um ativo financeiro \cite{ROLNIK2015}.

A produção de moradias, portanto, possibilita a criação de poupança interna de longo prazo e de uma forte geração de emprego e renda, por meio do aquecimento de toda a cadeia produtiva da construção civil. Além do impacto social que é o resgate da dignidade da pessoa humana através da moradia digna.

Frequentemente, estudos sobre o mercado de moradias tem sido analisado pela academia por meio de modelos de equilíbrio parcial ou adotando uma abordagem espacial por meio de modelos de preços do tipo hedônicos que levam em consideração essencialmente fatores microeconômicos. Nesse sentido, o mercado moradias é apresentado pelo viés da heterogeneidade, tanto da oferta como da demanda, onde os preços finais do tipo hedônico são decompostos como função de suas características intrínsecas, localização e amenidades urbanas \cite{Lancaster1966}.

Assim, temas que permeiam o mercado de moradias são apresentadas pela academia de forma autônoma, onde por um lado teríamos questões ligadas a politica habitacional como sendo um problema de agenda dos governos que buscam equacionar oferta e demanda no campo da moradia, seja de forma direta via programas habitacionais ou de forma indireta via melhora do ambiente de negócio, por outro lado, no campo da política monetário, teríamos a política creditícia de responsabilidade do Banco Central (BACEN), que teria como objetivo controlar a quantidade de moeda em circulação, o volume de crédito e as taxas de juros.  O que daria as condições para financiamento imobiliário de curto e longo prazo.

% falar sobre o sentido amplo de estudos na economia
Todavia, deve-se avançar na produção científica desses temas no qual levem em consideração a complexidade intrínsecas desses problemas, onde uma abordagem complexa da ciência deve ser tomada como referência teórica e metodológica.

% falar sobre a economia complexa.
Dessa forma, levar em consideração a economia e o mercado de moradias, em particular, como um sistema complexo é buscar tratar características como auto-organização, não-linealidade, redes de interação, \textit{feedbacks} positivos, dependência de trajetória além de emergência de padrões não apenas por meio de agregação (macroeconomia) ou por meio de agentes representativos (microeconomia). Mas, tratar os problemas relacionados com o mercado de moradias como uma metodologia que busque investigar como os agentes interagem ao longo do tempo em diferentes níveis de agregação e interação, ou seja, como as estruturas individuais influenciam níveis intermediários e hierárquicos (\textit{botton-up}) formando um todo orgânico de forma iterativa que não converge, obrigatoriamente, para o estado estacionário ou equilibrado \cite{2006TesfatsionJudd}.

É preciso, portanto, que estudos sobre o mercado de moradias que levem em consideração processos emergentes em um contexto de uma economia complexa onde os resultados de curto e longo prazo irão depender da forma como esses agentes se comportam  e interagem ao longo do tempo. No entanto, estudos que tratem sob esse abordagem ainda são incipientes no Brasil.
%citar estudos sobre o tema no Brasil.

Desse modo, é importante pôr em relevo a hipótese de que o funcionamento do mercado de moradias brasileiro, sob um capitalismo financiado, emerge uma estrutura social com elevada concentração da riqueza imobiliária onde um conjunto de famílias concentra a maior parte dessa riqueza em detrimento de um conjunto muito maior de famílias \cite{AALBERS2015}. Em outras palavras, em uma estrutura capitalista sob regime de acumulação financeira \cite{Aglietta2004financeirizacao, Orlean2006financeirizacao} o funcionamento do mercado de moradias resulta uma trajetória desigual.

Essa análise crítica ganha ainda mais importância quando levamos em consideração, em um contexto mundial, o expressivo crescimento do capitalismo financeirizado que podem ser expressos em termos objetivos no estoque de riqueza financeira mundial que no início da década de 1980 até os antecedentes da grande crise financeira de 2008, apresentou um crescimento na ordem de 1.292\% (13,9 vezes), enquanto que, no mesmo período, o PIB mundial apresentou um crescimento 314\% (4,1 vezes) \cite{Paulani2009a}.

Podemos acrescentar, para o caso específico do Brasil, o elevado deficit habitacional que tem persistido mesmo em um período onde houve elevada produção de moradias (2009-2015). Para a \citeonline{FJP2018} há no Brasil um deficit estimado de 6,4 milhões de moradias em 2015, correspondendo a 9,3\% do total de domicílios naquele ano. Metade do deficit é representado pelo \textit{"ônus excessivo com aluguel"}, ou seja, são famílias urbanas com renda familiar de até três salários mínimos que despendem mais de 30\% de sua renda com aluguel. Todavia, o que chama atenção é a elevada quantidade de imóveis vagos que segundo dados da PNAD de 2015, o Brasil tem 7,9 milhões de domicílios vagos, dos quais 80,3\% estão localizados em áreas urbanas. Desse montante, 6,9 milhões estão em plenas condições de serem habitadas \cite{ibge2015b}. O que demostra, em parte, as grandes distorções presente nesse mercado.

% ÓTICA MACRO
O que por outro lado, em uma perspectiva macroeconômica o mercado de moradias brasileiro demonstrou um importante papel pro-cíclico no movimento de crescimento observado do início dos anos 2000-2008 e anti-cíclico por conta da crise financeira mundial de 2008, apresentando crescimento consistente em diversos indicadores como, por exemplo, produção, emprego e volume de crédito para o setor. 
% -----------------------------------------------------------------
% atenção buscar autores que tratam sobre o papel pro e anticiclico
% -----------------------------------------------------------------

% ÓTICA MACRO - PRODUCAÇÃO
Em termo relativos na participação no valor adicionado bruto (a preços básicos) do Brasil, no que se refere a Indústria da Construção Civil juntamente com os serviços ligados as Atividades Imobiliárias, entre os anos de 2000 a 2015, apresentaram uma taxa média de 7,5\% de participação, o que equivale a cerca de R\$ 196 bilhões por ano, a preços constantes \footnote{Informações segundo as Contas Nacionais disponibilizado pelo Instituto Brasileiro de Geografia e Estatística - IBGE para vários anos. Ver site: \url{https://downloads.ibge.gov.br/downloads_estatisticas.htm}. Acesso em: 12 de Out. de 2018.}.

% ÓTICA MACRO - EMPREGO
No que diz respeito ao volume de emprego representou de 2006 a 2013, período de maior expansão, uma criação líquida de emprego formal no montante de 700 mil postos de trabalho o que representou 5,2\% do total de empregos formais gerado no Brasil naquele período. E como trata-se de um setor pro-cíclico, no período que inicia a crise econômica brasileira de 2014 até 2017 houve uma redução de 450 mil trabalhadores formais o que representa cerca de 21\% da redução de vagas no Brasil \footnote{Cadastro Geral de Empregados e Desempregados – CAGED disponibilizados pelo Instituto Brasileiro de Geografia e Estatística - IBGE para vários anos. Ver site: \url{https://downloads.ibge.gov.br/downloads_estatisticas.htm}. Acesso em: 12 de out. de 2018.}.

% ÓTICA MACRO - CRÉDITO
Já em termos de crédito imobiliário apresentou também uma forte expansão para setor. Saindo de 2007, com 1,8\% em proporção do PIB e saltando para cerca de 9,7\% em 2017, todavia, apesar do grande crescimento para o período analisado ainda se mostra muito a baixo da média de diversos outros países, mesmo entre os emergentes \cite{Cerutti2017, Otto2015}.

Nesse sentido, podemos observar que o mercado de moradias representa uma grande porcentagem da riqueza e da produção macroeconômica revelando um papel importante na dinâmica global do país em geral. Para \citeonline{Lucena2009}, a produção de moradias apresenta uma vantagem endógena adicional ao desenvolvimento do país, pois ao utilizar na maior parte do processo produtivo fatores de produção do próprio país, geraria um desenvolvimento mais harmônico.


%% IMPORTÂNCIA DE ESTUDAR
Dessa forma, entender empiricamente os fenômenos emergentes a partir de dinâmicas não-lineares por conta da interação entre os agentes em uma rede dispersa como é o mercado de moradias poderá levar a um melhor entendimento e uma atuação mais eficiente do ponto de vista econômico e social, tanto por parte do governo, das empresas privadas (bancos e construtoras) bem como das famílias, podendo deste modo, maximizar o bem estar social gerados pela produção e consumo dos imóveis residenciais no país.

Em termos metodológicos, investigar o mercado de moradias brasileiro em um perspectiva Complexa da Ciência nos faz repensar a própria separação metodológica entre fatores micro ou macroeconômicos como formas de construção do conhecimento em Economia \cite{Berghgowdy2003, DurantonPuga2004}

Logo, avançar na pesquisa sobre o mercado de moradias para além da separação entre microestrutura e macroestrutura, mas de evidenciar o elemento complexo que se apresenta quando há o efeito de uma sobre a outra, cujo equilíbrio não seria uma vetor constante do sistema nos tratá \textit{insights} importantes que a estática comparada não é capaz de responder, qual seja, quais são os efeitos dinâmicos do mercado de moradias urbano brasileiro decorrentes de sua estrutura interna sob um capitalismos liberado pelas finanças?

%% OBJETIVOS DO ESTUDO
A proposta principal desse projeto pesquisa, portanto, será o de desenvolver analiticamente um Modelo Baseado em Agentes para o mercado de moradias brasileiro tomando como referência empírica as nove principais regiões metropolitanas brasileira e que leve em consideração os dados reais como forma de \textit{calibração} e \textit{verificação} das simulações dos modelos desenvolvidos.  Isto é, ancorado em uma base empírica como forma de capturar fatos estilizados sobre o mercado em estudo, de modo a possibilitar o desenvolvimento de um Modelo Baseado em Agentes que seja capaz de produzir flutuações endógenas e irregulares, de modo que possa integrar em plano teórico e de modelagem tanto o ciclo como a tendência de longo prazo e seus possíveis resultados.





