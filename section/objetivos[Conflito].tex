\section{Objetivos}

A proposta desse projeto de pesquisa é o de elaborar três artigos que tratem sobre a dinâmica recente do mercado habitacional brasileiro em uma perspectiva tanto macroeconômica em um primeiro nível, como em uma perspetiva microeconômica em um segundo nível e, por fim propor, com base nos parâmetros desenvolvimento nos artigos anteriores, um modelo macrodinâmico que esteja microeconomicamente fundamentado para o mercado habitacional brasileiro. Desse forma, além do objetivo geral, apresentado abaixo, que dará sustentação geral aos três artigos, cada objetivo especifico apresentado trata-se do objetivo que cada artigo buscará desenvolver.

\subsection{Objetivo Geral}

Investigar quais foram os determinantes macro e microeconômicos que possibilitam a formação do ciclo de negócios do setor habitacional brasileiro e propor um modelo macrodinâmico que possa captar os movimentos, no plano teórico e modelístico, da formação do ciclo bem como apontar as tendências de longo prazo. 

\subsection{Objetivos Específicos}

\begin{itemize}
	\item Apresentar e discutir, a partir da literatura nacional e internacional, os avanços, as vantagens e os limites da utilização da modelagem e simulação em Economia e de sua aplicabilidade para investigar o Mercado Habitacional; % o que é? como funciona? principais tipos? ABM?
	\item Identificar empiricamente os determinantes macroeconômicos do Mercado Habitacional brasileiro no período recente (2003-2015), bem como seus efeitos dinâmicos de uma choque da política monetária tendo como foco construir fatos estilizados para o Mercado Habitacional que leve em consideração os fatores \textit{Macroeconômicos};
	\item Apontar, em uma perspectiva \textit{Microeconômica}, quais são os resultados na estrutura de gastos e endividamento das famílias que vivem nas zonas urbanas das regiões metropolitanas do Brasil por conta da utilização do crédito imobiliário com forma de aquisição do imóvel tendo como foco a construir fatos estilizados como elementos de verificação e calibração do modelo que será proposto ;
	\item Propor um Modelo Baseado em Agentes para o Mercado Habitacional Brasileiro que capture, teórica e modelisticamente, as flutuações do ciclo de negócios e sua tendência de longo prazo no qual seja resolvido por meio dos dados históricos reais como forma de verificação e calibração.  
\end{itemize}