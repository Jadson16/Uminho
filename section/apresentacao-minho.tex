\section{Apresentação}

%%Introdução
%Venho desenvolvendo minha tese de Doutoramento junto ao Programa de Pós-Graduação em Economia da Universidade Federal do Pará - PPGE/UFPA (Brasil) e após o cumprimentos de todos os créditos referentes as disciplinas obrigatórias e optativas estou na fase de elaboração do projeto e efetivo desenvolvimento da tese.

% Objetivo institucional

Venho através deste projeto de pesquisa formalizar um vínculo institucional do tipo Estágio Avançado Doutoral, junto a Escola de Economia e Gestão da Universidade do Minho - EEG/UMinho (Portugal), para que seja desenvolvido parte dos trabalhos do discente-pesquisador.


% Objetivos do estudo
De forma resumida, conforme detalhado no \nameref{sec:plano-invetigação}, o objetivo geral da pesquisa é o de propor um Modelo Baseado em Agentes ou \textit{Agent-Based Model} - ABM para o mercado de moradias urbanas das regiões metropolitanas do Brasil que seja calibrado como os dados reais desse mercado para verificação, de tal modo que possa integrar, no plano teórico e modelístico, as dinâmicas de formação do ciclo, bem como a tendência de longo prazo.

% Metodologia
Desse modo, na construção do modelo utilizaremos a calibração via dados observáveis como método de verificação do ABM. Conforme apresentado na \nameref{sec:proposta-metodologica}, iremos desenvolver o modelo em três etapas, onde cada etapa equivale a produção de um artigo científico. Sendo assim, a primeira etapa será a construção de fatos estilizados do mercado de moradias urbanas em um nível Microeconômico, tendo como foco a estrutura de endividamento via mercado de crédito; a segunda etapa seria a elaboração de fatos estilizados em um nível Macroeconômico com foco na formação do preço decorrente das variáveis macro-fundamentais e por fim, o modelo propriamente dito calibrado com os dados e fatos estilizados das etapas apresentadas anteriormente.

%falar que busco alguem que me auxilie, a principio, no desenvolvimento desses fatos estilidados sobre o mercado de moradias
Como nesse primeiro momento buscaremos desenvolver os artigos referentes as etapas um e dois, isto é, ancorado em uma base empírica como forma de capturar fatos estilizados sobre o mercado de moradias, busco junto a esse respeitado Programa Pós-Graduação um orientador acadêmico para que em conjunto com o orientador da minha Pós-Graduação de origem possa me acompanhar no desenvolvimento de um dos artigos pretendidos. tests tstes tesx 2 testes 2 teste 4

% Objetivo prático a UMinho
Nesses termos, a materialização e relatório final do Estágio Avançado Doutoral terá como formato o desenvolvimento de um \textit{paper}, em nível de publicação em revista indexada, no qual fará parte do resultado final da Tese de Doutoramento.