\section{Plano de Redação}

\subsection{Determinantes Macroeconômicos}

RESUMO:

O mercado habitacional tem sido geralmente analisado em modelos de equilíbrio parcial, ou adotando uma abordagem espacial/geográfica por meio de modelos de preços hedônicos. O objetivo desta tese é propor um modelo para o setor habitacional brasileiro em um contexto macroeconômico, onde busca verificar para o caso das regiões metropolitanas brasileira a tese levantada por CITAR - Aalbers, 2016 de que o mercado habitacional tem uma 

de um modelo macroeconômico em um arcabouço pós-keynesiano para toda a economia observando as relações microdinâmicas e com todas as principais relações entre os mercados real e financeiro devidamente explicadas e com forte vínculo com as contas nacionais. O modelo geral será desenvolvido na tradição de modelos consistentes de fluxo dinâmico de estoque ou Stock Flow Consistente Model (ver, por exemplo, Godley e Lavoie, 2007).

Tendo como objetivo principal verificar a hipótese de que o mercado imobiliário teria uma forte determinação 

\subsection{Determinante do Preço dos Imóveis das Regiões Metropolitanas no Brasil}

Busca 

\subsection{Avaliação de impacto do programa de crédito minha casa minha vida: }

\subsection{Modelagem SFC}