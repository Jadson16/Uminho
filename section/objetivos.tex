\section{Objetivos}


\subsection{Objetivo Geral}

Propor um Modelo Baseado em Agentes para o mercado de moradias urbanas das regiões metropolitanas do Brasil que seja calibrado com os dados reais desse mercado para verificação, de tal modo que possa integrar, no plano teórico e modelístico, as dinâmicas de formação do ciclo econômico, bem como da tendência de longo prazo.

\subsection{Objetivos Específicos}

\begin{itemize}
	\item Apresentar e discutir, a partir da literatura nacional e internacional, os avanços, as vantagens e os limites da utilização da modelagem e simulação em Economia e de sua aplicabilidade para investigar o mercado de moradias  brasileiro; 
	%---
	% Como a introdução ao trabalho % o que é? como funciona? principais tipos? ABM?
	
	\item Obter, em uma perspectiva \textit{Microeconômica}, fatos estilizados por meio das séries históricas reais como elementos de verificação e calibração do modelo que será proposto. Assim, buscaremos apontar quais são os resultados na estrutura de gastos e endividamento das famílias que residem na zona urbana das regiões metropolitanas do Brasil devido a utilização do crédito imobiliário;
	
	\item Buscar fatos estilizados para o mercado moradias que leve em consideração os fatores \textit{Macroeconômicos}. Isto é, identificar empiricamente os determinantes macroeconômicos do mercado de moradias brasileiro no período recente (2003-2015), bem como os efeitos dinâmicos de uma choque da política monetária;
	
	\item Propor um Modelo Baseado em Agentes para o Mercado de Moradias Brasileiro que integre, teórica e modelisticamente, as flutuações do ciclo de negócios e sua tendência de longo prazo no qual seja resolvido por meio dos dados históricos reais como forma de verificação e calibração.
\end{itemize}

